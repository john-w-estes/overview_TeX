%------------------------------------------------

\section{Introduction to 3D Printing}\index{Introduction to 3D Printing}\label{3dprint}

Becoming a skilled printer take time, patience, and a lot of mistakes! In this section, we will not endeavor to take a deep dive into all aspects of 3D printing, but instead we will give a quick overview of this fun, applicable, and quickly-growing field of technology.

\subsection{FDM Printers}

Fused deposition modeling (FDM) printers work by taking thermoplastic material at a high temperature and extruding it through a nozzle onto a bed layer by layer. Over time, these layers stack to form a solid, 3-dimensional object. FDM printing is the most common and accessible way to 3D-print. 

\noindent FDM materials vary in strength and flexibility. At Belhaven, we typically print in PLA filament, a cheap and dependable material. Other filaments include ABS (for pieces needed higher temperature tolerance), TPU (super-flexible, rubbery material), nylon, and filaments with additives such as wood, bronze, carbon fiber, and glow-in-the-dark particles.

\begin{remark}
An easy to way to think of FDM printing is like the reverse of erosion. Instead particles slowly being pulled away, they are slowly added.
\end{remark}

\begin{figure}[h]
    \centering
    \includegraphics[width=3in]{Pictures/exercise pics/prusa.jpg}
    \caption{The Prusa i3 MK3S is a FDM printer}
    \label{fig:prusai3}
\end{figure}

\noindent You will learn to use a FDM printer in this course.

\subsection{SLA Printers}

Stereolithography (SLA) printers actually proceeded FDM printing, being invented in the 1980's \cite{formlabs}, but has not be as publicly accessible until recent years. SLA printers use lasers to cure resin into a hardened plastic.

\begin{figure}[h]
    \centering
    \includegraphics[width=3in]{Pictures/exercise pics/peopoly.png}
    \caption{The Peopoly Phenom is a SLA printer}
    \label{fig:prusai3}
\end{figure}

\subsection{The Software of Printing}

One of the exciting capabilities of 3D printing is that it takes something digital and manifests it physically. This process involves 
\begin{enumerate}
    \item finding a digital object in the form of an STL file,
    \item placing this object along with print settings on a canvas in slicing software,
    \item slicing a printable file (Slicing is the process of writing a program for the printer to print a specific object.),
    \item transferring the file to the printer.
\end{enumerate}

\noindent This may sound like a complicated process, but these steps happen pretty quickly in practice. We will learn each step in this chapter.

\subsection{Printers at Belhaven University}

The Belhaven Maker Campus features several printing labs with both FDM and SLA printers. 

\noindent In Fitzhugh Hall, you can find a variety of printing options ranging from the tiny Monoprice Mini (120 mm $\times$ 120mm $\times$ 120mm) to the enormous Formbot Raptor 2 (400mm $\times$ 400mm $\times$ 700mm).

Additionally, you can find
\begin{itemize}
    \item Prusa i3 MK3S 
    \item Prusa i3 MK3S with MMU2 (prints up to 5 colors)
    \item Prusa Mini
    \item MakerGear M3 IE (two extruders running independently)
    \item Artillery Sidewinder X1
    \item Prusa SL1S (SLA printer)
\end{itemize}

\noindent Belhaven's Art and Design Department houses another 3D printing lab that includes

\begin{itemize}
    \item Flashforge Finder
    \item Lulzbot Workhorse with Pallette 2 (multiple colors)
    \item Peopoly Phenom (SLA printer)
\end{itemize}

\noindent In total, members the Maker Campus have plenty of opportunities to print from early entry printing all the way to advanced printing projects.
