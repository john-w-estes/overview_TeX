%------------------------------------------------

\section{HOMEWORK: The Four Color Theorem}\index{HOMEWORK: The Four Color Theorem}

Consider the map  of the United States in figure \ref{map}. You are challenged to color the map with the rule that bordering states cannot have the same color. Overall this is pretty easy, BUT we also want to use the fewest colors possible. How many colors do you need to do this?

\begin{exercise}
Color the map in the fewest colors possible, where neighboring states do not share a color.
\end{exercise}

\begin{figure}[h]
    \centering
    \includegraphics[width=\textwidth]{Pictures/exercise pics/US_map.PNG}
    \label{map}
\end{figure}

\noindent This is actually a graph theory question. In graph theory, mathematicians are often curious about the minimum number of colors required to color the vertices so that neighboring vertices do not share the same color. 

\newpage

\begin{exercise}
Draw a graph where the vertices can be colored in 2 colors.
\end{exercise}

\vspace{3in}

\begin{exercise}
Draw a graph in which 3 colors are needed to color the graph.
\end{exercise}

\vspace{3in}

\begin{exercise}
How about a graph that requires 5 colors?
\end{exercise}

\vspace{3in}

\subsection{The Four Color Theorem}

Remember Numberphile? They have yet another excellent video about the incredible Four Color Theorem.

\begin{exercise}
Watch the Numberphile video “The Four Color Map Theorem”, \cite{numberphile_fourcolor}
\end{exercise} 
\footnotesize{https://www.youtube.com/watch?v=NgbK43jB4rQ}

\newpage

\begin{exercise}
How many colors do we need to color any map so that neighboring countries do not share the same color?
\end{exercise}

\blanks

\begin{exercise}
When was the Four Color Theorem finally solved? Who proved it?
\end{exercise} 

\blanks

\begin{exercise}
When phrasing the question in terms of graph theory. How do we represent countries? 
\end{exercise}

\blanks

\begin{exercise}
What is the benefit of phrasing the Four Color Theorem as a graph theory problem?
\end{exercise}

\blanks
\blanks

\begin{exercise}
Give a brief explanation of the argument that map needs no more than seven colors.
\end{exercise}

\blanks
\blanks

\begin{exercise}
Why was Appel and Haken’s proof controversial?
\end{exercise}

\blanks
\blanks





