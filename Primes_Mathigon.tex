%------------------------------------------------

\section{Divisibility and Primes in Mathigon}\index{Divisibility and Primes in Mathigon}

Mathigon is a fantastic and quickly-growing resource for mathematics. There goal is offer `free tools, courses and manipulatives to make online learning more interactive and engaging than before. We will take now look at one of Mathigon's courses ``Divisibility and Primes".

\noindent From mathigon.org, navigate to Courses $\rightarrow$ Foundations $\rightarrow$ Divisibility and Primes. Answer the following questions.

\subsection{Divisibility}

\begin{exercise}
How do you know a number is divisible by the following numbers?
\end{exercise}

\begin{enumerate}
    \item 2: 
    \item 5:
    \item 4:
    \item 8:
    \item 3:
    \item 9:
    \item 6:
\end{enumerate}

\subsection{Famous Prime Results}

\begin{exercise}
By use of the Sieve of Eratosthenes, how many prime numbers are less than 100?
\end{exercise}

\vspace{1in}

\begin{exercise}
How many prime numbers are there? Who proved this result?
\end{exercise}

\vspace{1in}

\begin{exercise}
Is 17485637 a prime number?
\end{exercise}

\vspace{1in}

\begin{exercise}
Give an 8-digit prime number.
\end{exercise}

\newpage

\begin{exercise}
What is the Goldbach Conjecture? 
\end{exercise}

\blanks

\begin{exercise}
Express 1763858 as a sum of two prime numbers.
\end{exercise}

\vspace{1in}

\begin{exercise}
How is the Riemann Hypothesis connected to prime numbers?
\end{exercise}

\blanks
\blanks

\subsection{LCM and GCD}

\begin{exercise}
Find the lcm$(44, 92)$.
\end{exercise}

\vspace{1in}

\begin{exercise}
Find the $\gcd(192, 96)$.
\end{exercise}

\vspace{1in}

\begin{exercise}
RSA Cryptography depends on what key mathematical concept?
\end{exercise}

\vspace{1in}

