\section{Geogebra and Linear Transformations}\index{Geogebra and Linear Transformations}

 
 \noindent In this lab, we will use Geogebra, a fantastic tool for visualizing and animating mathematical concepts, to explore applications of linear transformations in the plane. Before getting started, visit geogebra.org and either download Geogebra Classic 6 (found under App Downloads) or use the web applet. Downloading the app is recommended, and in either case, be sure to set up a free Geogebra account.

% Talking about Linear Transformations ------------------------------------------------- 
\subsection{Linear Transformations and the Standard Matrix}

Believe it or not, matrix multiplication happens before our eyes all of the time. Many computer graphics and graphic design elements rely on vector graphics and something called ``linear transformation".
For example, the linear transformation that maps vector $\left[\begin{array}{c}
x\\
y
\end{array}\right]$ to vector $\left[\begin{array}{c}
ax + by\\
cx + dy
\end{array}\right]$ can be characterized by its standard matrix $A = \left[\begin{array}{cc}
a & b\\
c & d
\end{array}\right]$ as seen by the following calculation.

\begin{example}
\[\left[\begin{array}{cc}
a & b\\
c & d
\end{array}\right] \hspace{1em} \left[\begin{array}{c}
x\\
y
\end{array}\right] = \left[\begin{array}{c}
ax + by\\
cx + dy
\end{array}\right]\]
\end{example}

\noindent The linear transformation $T: \mathbb{R}^2 \rightarrow \mathbb{R}^2$ with standard matrix $A$ can be visualized geometrically by the way it transforms an object. This is what we want to do in this section.

% Initial Geogebra Setup ------------------------------------------------- 
\subsection{Geogebra Setup}

\begin{exercise}[Drawing our Polygon]
Once you have Geogebra opened, place five to ten points into the plane with each point within the rectangle $[-5,5] \times [-5,5]$. You can do this by simply selecting ``point" from the ``Tools" menu and clicking where in the plane you want those points.
\end{exercise}

\noindent In the Algebra menu, notice that each of these points has a name and a coordinate. If we use the ``Move" tool, we can drag a point in the plane, and its coordinate value updates dynamically. \\

\noindent Now select ``Polygon" from the ``Polygons" menu, and connect your dots to form a polygon in the plane. This is the shape we will manipulate. Draw your polygon in the space below. (You may hide the labels and change colors within the Settings menu if you would like.)

\begin{center}
\includegraphics[width=4in]{Pictures/exercise pics/geogebragrid.PNG}
\end{center}

\begin{exercise}[Defining our Matrix]
In the Algebra menu, type ``matrix = $\{ \{ 1, 0 \}, \{ 0, 1 \} \}$". The output should be $I_2$, the $2 \times 2$ identity matrix.
\end{exercise}

\begin{exercise}[Adding our vectors]
Now that we have our polygon and a matrix, find the ``Vectors" tool under the ``Lines" menu, and add vectors that start at the origin and end at each point of our polygon. Check to see in the Algebra menu that each vector is of the form Vector(G, A), meaning that each vector is defined by two points.
\end{exercise}

\begin{exercise}[Matrix Multiplication]
In the Algebra menu, multiply each vector Vector(G, A) by our matrix. We do this by placing our cursor before ``Vector" and typing ``(matrix Vector(G,A))" Do this for each vector.
\end{exercise}

\begin{center}
\includegraphics[width=2in]{Pictures/exercise pics/geogebratransforms1.PNG}    
\end{center}

\noindent By multiplying each vector by the identity matrix, we have performed a transformation that does not move anything. We will change ``matrix" in the next section to perform different transformations on our polygon.\\

\begin{exercise}[Drawing our New Polygon]
For our visual to work correctly, we must hide our polygon and draw a new one. To do so, complete the following.
\end{exercise}

\begin{enumerate}
    \item In the Algebra menu, click the colored circle on the left margin beside each point and the polygon to hide those objects. You should only see your vectors.
    \item Add new points at the end of each vector. Check in the Algebra menu that each point is defined by the vector like in the picture below.
    \begin{center}
    \includegraphics[width=2in]{Pictures/exercise pics/geogebratransforms2.PNG}    
    \end{center}
    \item Draw a polygon connecting your new points. It should look identical to your original, hidden polygon.
    \item Lastly, hide your vectors and unhide your original polygon. 
\end{enumerate}

\noindent We are now ready to perform some transformations!

 
% Different Matrices ------------------------------------------------- 
\subsection{Testing Different Matrices}

\noindent Now that we have our Geogebra configuration correct, all we need to do is to change our matrix to perform a transformation. Redefine your matrix with the following matrices. Then draw your result and describe what transformation that matrix applied.

\begin{exercise} 
Update your matrix to the following.
$\left[\begin{array}{cc}
-1 & 0\\
0 & 1
\end{array}\right]$
\end{exercise}

\begin{center}
\includegraphics[width=3in]{Pictures/exercise pics/geogebragrid.PNG}\end{center}

\noindent What type of transformation is this?  \rule{5cm}{0.15mm} \\

\newpage 

\begin{exercise}
Update your matrix to the following.
$\left[\begin{array}{cc}
1 & 0\\
0 & -1
\end{array}\right]$
\end{exercise}

\begin{center}
\includegraphics[width=3in]{Pictures/exercise pics/geogebragrid.PNG}\end{center}

\noindent What type of transformation is this?  \rule{5cm}{0.15mm} \\

\begin{exercise}
Update your matrix to the following
$\left[\begin{array}{cc}
2 & 0\\
0 & 2
\end{array}\right]$
\end{exercise}

\begin{center}
\includegraphics[width=3in]{Pictures/exercise pics/geogebragrid.PNG}\end{center}


\noindent What type of transformation is this?  \rule{5cm}{0.15mm} \\

\begin{exercise}
Update your matrix to the following
$\left[\begin{array}{cc}
2 & 0\\
0 & 1
\end{array}\right]$
\end{exercise}

\begin{center}
\includegraphics[width=3in]{Pictures/exercise pics/geogebragrid.PNG}\end{center}

\noindent What type of transformation is this?  \rule{5cm}{0.15mm} \\

\begin{exercise}
Update your matrix to the following
$\left[\begin{array}{cc}
0 & 1\\
1 & 0
\end{array}\right]$
\end{exercise}

\begin{center}
\includegraphics[width=3in]{Pictures/exercise pics/geogebragrid.PNG}\end{center}

\noindent What type of transformation is this?  \rule{5cm}{0.15mm} \\

\begin{exercise}
Update your matrix to the following
$\left[\begin{array}{cc}
1 & 3\\
0 & 1
\end{array}\right]$
\end{exercise} 

\begin{center}
\includegraphics[width=3in]{Pictures/exercise pics/geogebragrid.PNG}\end{center}

\noindent What type of transformation is this?  \rule{5cm}{0.15mm} \\

% Rotating ------------------------------------------------- 
\subsection{Rotating our Polygon}

\noindent Now we will animate a rotation of our object. \\

\begin{exercise}
Select ``slider" from the Basic Tools menu and add it to the plane. A menu should pop up. Take note of the slider name, and set the Min to be -2pi and the Max to be 2pi. For our example, the slider name is ``p".
\end{exercise}

\begin{center}
\includegraphics[width=2in]{Pictures/exercise pics/geogebratransforms3.PNG}   
\end{center}

\begin{exercise}
Now redefine matrix as $\left[\begin{array}{cc}
\cos(p) & \sin(p)\\
-\sin(p) & \cos(p)
\end{array}\right]$. Be sure the inserted angle is the name of your slider.
\end{exercise}

\noindent What is happening to your polygon as you toggle your slider? 

\blanks

\begin{exercise}
Under the Settings menu, select ``Share". Copy the link to your Geogebra activity and post it in Canvas under the assignment ``Geogebra Transformation".
\end{exercise}
