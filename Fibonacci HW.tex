%------------------------------------------------

\section{HOMEWORK: The Fibonacci Sequence}\index{HOMEWORK: The Fibonacci Sequence}

The Fibonacci Sequence is probably the most famous of recursively defined sequences. A recursively defined sequence is a sequence in which the next term in the sequence is given by a formula of previous term in the sequence. 

\noindent The Fibonacci Numbers are defined as follows:

\begin{center}
    \begin{equation*}
        F_1 = 1, \; F_2 = 1 \; \text{ and } F_n=F_{n-1}+F_{n-2} {for } n\geq 3.
    \end{equation*}
\end{center}

\noindent So then the Fibonacci Sequence can be listed as 1, 1, 2, 3, 5, $\ldots.$

\begin{exercise}
Write out the first 15 Fibonacci numbers.
\end{exercise}

\begin{multicols}{2}
\begin{enumerate}
    \item $F_1 =$
    \vspace{.25in}
    \item $F_2 =$
    \vspace{.25in}
    \item $F_3 =$
    \vspace{.25in}
    \item $F_4 =$
    \vspace{.25in}
    \item $F_5 =$
    \vspace{.25in}
    \item $F_6 =$
    \vspace{.25in}
    \item $F_7 =$
    \vspace{.25in}
    \item $F_8 =$
    \vspace{.25in}
    \item $F_9 =$
    \vspace{.25in}
    \item $F_{10} =$
    \vspace{.25in} 
    \item $F_{11} =$
    \vspace{.25in}
    \item $F_{12} =$
    \vspace{.25in}
    \item $F_{13} =$
    \vspace{.25in}
    \item $F_{14} =$
    \vspace{.25in}
    \item $F_{15} =$
\end{enumerate}	
\end{multicols}

\subsection{Number Formulas with Fibonacci Numbers}

There are a lot of interesting Fibonacci number formulas we can prove with basic proof techniques. We won't worry about proofs here (leave that for Discrete Math and Proof Exposition), but we will look at a few interesting examples.

\begin{theorem}\label{fib1}
The Fibonacci Sequence obeys $F_{n+1}^2 - F_{n+1}F_n- F_n^2 = (-1)^n$.
\end{theorem}

\begin{exercise}
Verify Theorem \ref{fib1} for $n = 8$.
\end{exercise}

\vspace{2in}

\begin{theorem}\label{fib2}
The Fibonacci Sequence obeys $F_1 + F_2 + F_3 + F_4 + \ldots + F_n = F_{n+2} - 1$.
\end{theorem}

\begin{exercise}
Verify Theorem \ref{fib2} for $n = 8$.
\end{exercise}

\vspace{2in}

\begin{theorem}\label{fib3}
The Fibonacci Sequence obeys $F_1 + F_3 + F_5 + F_7  + \ldots + F_{2n-1} = F_{2n}$.
\end{theorem}

\begin{exercise}
Verify Theorem \ref{fib2} for $n = 8$.
\end{exercise}

\vspace{2in}

\subsection{The Golden Rectangle}

\vspace{-.1in}

\noindent As we'll see, the Fibonacci Sequence appears in many surprising places. Two of its direct applications are the Golden Rectangle and Golden Spiral. Let's take a look.

\newpage 

\begin{exercise}
Use, as a reference, the article “Patterns in Natures: Where to Spot Spirals” to construct the Golden Rectangle and the Golden Spiral \cite{fibonacci}.
\end{exercise}
\footnotesize{https://www.scienceworld.ca/stories/patterns-nature-where-spot-spirals/}

\vspace{3in}

\begin{exercise}
Give an example of where the Golden Rectangle or the Golden Spiral appears in nature.
\end{exercise}

\blanks

\subsection{Fibonacci Numbers and Plants}

Youtuber, Vihart, has a great illustration of the Fibonacci Sequence's appearance among plants. 

\begin{exercise}
Watch Vihart's YouTube video ``Doodling in Math: Spirals, Fibonacci, and Being a Plant [1 of 3]" \cite{vihart}.
\end{exercise}
\footnotesize{https://youtu.be/ahXIMUkSXX0}

\begin{exercise}
After reading the Science World article and watching Vihart's video, write your thoughts about the Fibonacci sequence.
\end{exercise}

\blanks
\blanks

\subsection{More Fibonacci}

We have only explored a few places where the Fibonacci sequence appears in real life. Let's find some more.

\begin{exercise}\label{onlinefibonacci}
Look online to find 10 places you can find the Fibonacci sequence. Be sure to include the golden rectangle, the golden spiral, nature, and art in your list.
\end{exercise}

\begin{multicols}{2}
\begin{enumerate}
    \item \hspace{.5in}
    \vspace{.25in}
    \item \hspace{.5in}
    \vspace{.25in}
    \item \hspace{.5in}
    \vspace{.25in}
    \item \hspace{.5in}
    \vspace{.25in}
    \item \hspace{.5in}
    \vspace{.25in}
    \item \hspace{.5in}
    \vspace{.25in}
    \item \hspace{.5in}
    \vspace{.25in}
    \item \hspace{.5in}
    \vspace{.25in}
    \item \hspace{.5in}
    \vspace{.25in}
    \item \hspace{.5in}
\end{enumerate}	
\end{multicols}

\begin{exercise}
Choose one of the items in Exercise \ref{onlinefibonacci} and share in the Canvas discussion ``Fibonacci in the World". Be sure to include your source as well.
\end{exercise}






