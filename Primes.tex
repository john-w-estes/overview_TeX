%------------------------------------------------

\section{Primes}\index{Primes}

What are fundamental building blocks of the natural numbers? In other words, is there a set of numbers from which every other number can be derived? The principal way we provide an answer to these questions is with prime numbers.

\subsection{What is a prime?}
We say an integer $p$ is prime if (1) $p > 1$ and (2) the only numbers that divide $p$ are $1$ and $p$.
 
\subsection{Building blocks of the Integers}
We already mentioned that all numbers are derived from prime numbers, but what does that mean?

\noindent The Fundamental Theorem of Arithmetic states that every positive integer is a unique product of primes. You can imagine that every positive integer has a unique serial number generated by its prime factorizaton.

\begin{example}
Find the prime factorization of 48.

No problem, $48 = 2^4 \times 3$.
\end{example}

\begin{exercise}
Find the prime factorization of the following.
\end{exercise}

\begin{enumerate}
    \item $84$ 
    
    \vspace{1in}
    
    \item $1102$
    
    \vspace{1in}
    
    \item $50,260$
    
\end{enumerate}
  \vspace{1.5in}
  
\subsection{Divisors}
If the number $d$ appears in the prime factorization of $n$, then we say that $d$ divides $n$ (or that $d$ is a divisor of $n$.) So 8 is a divisor of 48 since $8 = 2^3$ is a part of the prime factorization of 48.

\noindent We define $d(n)$ to be the number of divisors of integer $n$.

\begin{exercise}\label{dn}
Find $d(n)$ for the following values of $n$.
\end{exercise}

\begin{enumerate}
    \item $n = 12$
    
    \vspace{1in}
    
    \item $n = 63$
    
    \vspace{1in} 
    
    \item $n = 60$
    
    \vspace{1in}
    
    \item $n = 23$
    
\end{enumerate}

\subsection{A formula for $d(n)$}
A big part of mathematics is discovering patterns. Exercise \ref{dn} is a good place to start to find a formula for $d(n)$. Where do we start though? Primes.

\begin{exercise}
Let $p$ be a prime number. Find $d(p)$.
\end{exercise}

\newpage

\begin{exercise}
What about $d(p^2)$? Start with an example if needed.
\end{exercise}

\vspace{2in}

\begin{exercise}
What about $d(p^{\alpha})$ for $\alpha \geq 1$? Start with an example if needed.
\end{exercise}

\vspace{2in}

\begin{exercise}
Let $p$ and $q$ be prime numbers. Find $d(pq)$.
\end{exercise}

\vspace{2in}

\noindent Maybe we're getting closer. Ask yourself why we've picked these examples. We started with a prime, $p$. Then we looked at $p$ at some power. After that, we looked at two primes being multiplied together.

\noindent Remember that the Fundamental Theorem of Arithmetic says that any positive integer is a unique product of primes. So if 
\begin{equation*}
    n = p_1^{\alpha_1}p_2^{\alpha2}\ldots p_k^{\alpha_k},
\end{equation*}

\noindent and we know how to figure out $d(p_1^{\alpha_1}p_2^{\alpha2}\ldots p_k^{\alpha_k})$, then we have figured out $d(n)$ for any positive integer!

\newpage 

\begin{exercise}
Suppose $ n = p_1^{\alpha_1}p_2^{\alpha2}\ldots p_k^{\alpha_k}$. Find a formula for $d(n)$.
\end{exercise}

