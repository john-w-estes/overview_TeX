%------------------------------------------------

\section{HOMEWORK: The Menger Sponge and the Mandlebrot Set}\index{HOMEWORK: The Menger Sponge and the Mandlebrot Set}

There are several famous fractals of interest. We already saw Sierpinski's Triangle. Another one of these is the Menger Sponge.

\begin{figure}[h]
    \centering
    \includegraphics[width=3in]{Pictures/exercise pics/menger.png}
    \caption{The Menger Sponge}
    \label{fig:menger}
\end{figure}

\subsection{A Few of My Favorite Spaces}

Mathematician Evelyn Lamb has written as an incredible series entitled ``A Few of My Favorite Space" for Scientific American covering different mathematical spaces.

\begin{exercise}
Read “A Few of My Favorite Spaces: The Menger Sponge” by Evelyn Lamb in Scientific American \cite{menger}.
\end{exercise}
{\footnotesize  https://blogs.scientificamerican.com/roots-of-unity/a-few-of-my-favorite-spaces-the-menger-sponge/.}

\subsection{A Few Questions}

After reading Lamb's article, address the following questions.

\begin{exercise}
How do you develop the Menger Sponge?
\end{exercise}

\blanks
\blanks

\begin{exercise}
What is the dimension of the Menger Sponge?
\end{exercise}

\blanks
\blanks

\begin{exercise}
What are your thoughts about the Menger Sponge?
\end{exercise}

\blanks
\blanks

\subsection{The Mandelbrot Set and Numberphile}

Another famous fractal is the Mandelbrot Set shown in Figure \ref{fig:mandelbrot}.

\begin{figure}[h]
    \centering
    \includegraphics[width=3in]{Pictures/exercise pics/mandlebrot.png}
    \caption{The Mandelbrot Set}
    \label{fig:mandelbrot}
\end{figure}

\noindent Numberphile is a popular math-related YouTube channel that covers a huge variety interesting high-level mathematics topics. Numberphile is a fantastic resource to mathematics students. We will look to Numberphile several times in this text.

\begin{exercise}
Watch Numberphile's video “The Mandelbrot Set”. \cite{numberphile_mandelbrot}
\end{exercise}
{\footnotesize https://www.youtube.com/watch?v=NGMRB4O922I\&t=473s } 

\begin{exercise}
Explain the process that determines whether or not a number will be in the Mandelbrot Set or not.
\end{exercise}

\blanks
\blanks


