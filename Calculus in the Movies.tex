%------------------------------------------------

\begin{figure}[h]
    \centering
    \includegraphics[width=\textwidth]{Pictures/exercise pics/nemo.PNG}
    \label{fig:nemo}
\end{figure}

\section{HOMEWORK: Calculus in the Movies}\index{HOMEWORK: Calculus in the Movies}

Using Infinitely close approximations displays power of calculus. Did you know that you can see calculus at work when you watch Finding Nemo?

\noindent In fact Pixar Studios continually conducts rigorous mathematical research. One of the mathematicians working for Pixar is Tony Rose, the first PhD in mathematics to win an Academy Award!

\subsection{Numberphile's Math and Movies}

Numberphile visited Pixar Studios to meet with Tony Rose to discuss some of the mathematics behind their movies.

\begin{exercise}
Watch Numberphile’s YouTube video "Math and Movies (Animation at Pixar)"\cite{numberphile_movies}. 
\end{exercise}
{\footnotesize https://youtu.be/mX0NB9IyYpU }

\begin{exercise}
Explain how the concept of a limit has helped Pixar animators.
\end{exercise}

\blanks
\blanks

\begin{exercise}
Why do newer animated movies look better than the pioneers of 3D animation like Toy Story and Ice Age?
\end{exercise}

\blanks
\blanks

\subsection{What about other Movies?}

It's easy to see the use of mathematical computation in animated films dependent on computer modeling, but what about other movies?

\begin{exercise}
Where do you think?
\end{exercise}

\blanks
\blanks


