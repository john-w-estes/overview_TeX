%------------------------------------------------

\section{HOMEWORK: Defining Dimension}\index{HOMEWORK: Defining Dimension}

We have now seen the dimension of fractals and of vector spaces. Mathematician and author, David Richeson, explored the concept of dimension further in amazing Quanta Magazine article: ``The Journey to Define Dimension" \cite{richeson}.


\subsection{The Journey to Define Dimension}

\begin{exercise}
Read David Richeson’s article ``The Journal to Define Dimension" from Quanta Magazine. 
\end{exercise}
{\footnotesize https://www.quantamagazine.org/a-mathematicians-guided-tour-through-high-dimensions-20210913/ }

\begin{exercise}
Describe what we would observe if a 4-dimensional sphere collided with our 3-dimensional space.
\end{exercise}

\blanks
\blanks

\begin{exercise}
List four people mentioned in this article and their contribution to the discussion of dimension.
\end{exercise}

\begin{enumerate}
    \item \hspace{1in} \\
    \item \hspace{1in} \\
    \item \hspace{1in} \\
    \item
\end{enumerate}

\begin{exercise}
Georg Cantor talked about the cardinality of infinity. What does he mean by ``cardinality”? Explain what Cantor was referring to when he said, ``I see it, but I do not believe it.”
\end{exercise}

\blanks
\blanks

\begin{exercise}
How did Hausdorff define dimension? What mathematical concept came from his definition?
\end{exercise}

\blanks
\blanks

\subsection{Quanta Magazine}

Quanta Magazine offers a lot of interesting high-level content in the sciences and mathematics.

\begin{exercise}
Find an article from Quanta Magazine that is interesting to you, and post a link to it in the Canvas discussino post ``Quanta Magazine".
\end{exercise}
