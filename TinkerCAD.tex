%------------------------------------------------

\section{TinkerCAD}\index{TinkerCAD}\label{tinkercad}

In general, designing 3D objects can be difficult. At times, it requires precision. Other times, it requires an artistic hand. Even though designing is not always quickly approachable, thankfully there is TinkerCAD.

\subsection{Thingiverse}

TinkerCAD, from Autodesk, is a free online design app. Designers can easily drag and drop manipulatives on a canvas to build 3D printable objects.

\begin{wrapfigure}{r}{0.25\textwidth} %this figure will be at the right
    \centering
    \includegraphics[width=.7in]{Pictures/exercise pics/tinkercad.png}
    %\caption{The Prusa i3 MK3S is a FDM printer}
    \label{fig:tinkercad}
\end{wrapfigure}

\begin{exercise}
Visit tinkercad.com, and setup an account.
\end{exercise}
{\footnotesize https://www.tinkercad.com/}

\noindent TinkerCAD has a quick tutorial to get students started designing quickly. You should go through the tutorial!

\begin{exercise}
Complete the ``Learning the Moves" and ``Camera Controls" tutorials in TinkerCAD.
\end{exercise}

\subsection{Blender}

TinkerCAD is the first place to look to become a designer, but there many others. Other popular design programs include

\begin{itemize}
    \item AutoCAD (by Autodesk)
    \item Inventor (by Autodesk)
    \item Solidworks
\end{itemize}

\noindent These products are professional, but not free. However, Blender is another, more sophisticated, design program that is completely free. If you are interested in taking your design skills deeper, you can can download Blender for free.
{\footnotesize https://www.blender.org/}



