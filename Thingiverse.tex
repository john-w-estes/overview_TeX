%------------------------------------------------

\section{Finding Things}\index{Finding Things}

What do you want to print? Do you have to design everything? Thankfully, no. Becoming a skilled designer takes time and experience, and many of us want to start 3D printing well before we can design our own things.

\noindent Luckily, there are some great resources where designers can share their designs and where we can download them to print ourselves.

\subsection{Thingiverse}

The quickest way to find good STL files is with Thingiverse, a social media platform for 3D printing enthusiasts to share their designs.

\begin{wrapfigure}{r}{0.25\textwidth} %this figure will be at the right
    \centering
   \includegraphics[width=.7in]{Pictures/exercise pics/Thingiverse_Logo_192x192.png}
%    \caption{The Prusa i3 MK3S is a FDM printer}
    \label{fig:prusai3}
\end{wrapfigure}


\begin{exercise}
Visit thingiverse.com, and setup an account.
\end{exercise}
{\footnotesize https://www.thingiverse.com/}

\begin{exercise}
Find something you think looks cool, and post a link to it under the Cool STL assignment found in Canvas.
\end{exercise}

\begin{exercise}
Find a thing from Thingiverse that you would like to print. This object must have only one STL file. Send a link to this thing to your instructor for approval. This will be your first print.
\end{exercise}

\subsection{Other Places for Things}

Thingiverse is not the only pace to find things. You can also find STL files at

\begin{itemize}
    \item Thangs {\footnotesize https://thangs.com}
    \item Cults3D {\footnotesize https://cults3d.com/en}
    \item Printables by Prusa {\footnotesize https://www.printables.com/en}
\end{itemize}

