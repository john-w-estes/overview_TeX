%------------------------------------------------

\section{Intro to Fractals}\index{Intro to Fractals}

A fractal is a mathematical set in which each subpart is statistically similar to the whole. We typically introduce fractals as a structure obtained through the process of iteration.

\noindent We will pull several interesting examples from Lippman's ``Math in Society" \cite{math_in_society}. The first example illustrates the process of fractal generation of iteration.

\begin{example}\label{fractal_exmample1}
Suppose that we start with a filled-in triangle. At each step of iteration, we connect the midpoint of each side and remove the middle triangle. We then repeat these steps.

\begin{figure}[h]
    \centering
    \includegraphics[width=3in]{Pictures/exercise pics/fractal example1.png}
    %\caption{Microsoft Teams}
    \label{fig:fractal1}
\end{figure}
\end{example}

\begin{definition}
The initial shape in a fractal is called the \textbf{initiator}, and the iterative action is known as the \textbf{generator}.
\end{definition}

\noindent In Example \ref{fractal_exmample1}, the black triangle is the initiator, and the iterative step is action of connecting the midpoints of each side and removing the middle triangle.

\newpage
 
\subsection{Playing with Fractals (Group Work)} 

Break into groups of three to complete the following exercises.


\begin{exercise}
Use the given initiators and generators to create the following iterative fractals. Draw the iterative steps 1 through 3.
\end{exercise}
\begin{figure}[h]
    \centering
    \includegraphics[width=1.5in,left]{Pictures/exercise pics/fractal exercise 1.png}
    %\caption{Microsoft Teams}
    \label{fig:fractalex1}
\end{figure}

\vspace{2in}

\begin{exercise}
\end{exercise}
\begin{figure}[h]
    \centering
    \includegraphics[width=1.5in,left]{Pictures/exercise pics/fractal exercise 2.png}
    %\caption{Microsoft Teams}
    \label{fig:fractalex2}
\end{figure}


\newpage

\begin{exercise}
\end{exercise}
\begin{figure}[h]
    \centering
    \includegraphics[width=1.5in,left]{Pictures/exercise pics/fractal exercise 3.png}
    %\caption{Microsoft Teams}
    \label{fig:fractalex3}
\end{figure}

\vspace{2in}

\subsection{Fraction Dimension}

We will discuss Dimension in more detail at a later time, but we all have some intuition of dimension.

\begin{figure}[h]
    \centering
    \includegraphics[width=2.5in]{Pictures/exercise pics/fractal example1.png}
    \caption{Fractal examples}
    \label{fig:fractal1}
\end{figure}

\begin{exercise}
As a class, give an example of a 1-dimensional object. How about a 2-dimensional object? A 3-dimensional object?
\end{exercise}

\noindent Those are easy, but what about the fractal known as Sierpinski's Triangle shown below?

\begin{figure}[h]
    \centering
    \includegraphics[width=2in]{Pictures/exercise pics/sierpinski.png}
    \caption{Sierpinski`s Triangle}
    \label{fig:sierpinski}
\end{figure}

\noindent It looks 2-dimensional, but is it? Consider the situation in Example \ref{fractal_exmample1}. 

\noindent Notice that the Step 1 figure has 1/3 the area of the generator figure, and the Step 2 figure has 1/3 the area of the Step 1 figure. Notice that if we keep going through this process, we end up with Sierpinski's Triangle... and how much area does it have? 2?

\begin{remark}
Fractals are interesting in that their measured dimension is not an integer! Fractal dimension is found by
\begin{equation*}
    D = \dfrac{\log(C)}{\log{S}},
\end{equation*} where it takes $C$ copies of the original shape to scale it by a factor of $S$.
\end{remark}


\begin{example}
The dimension of Sierpenski's Triangle is 

    \begin{equation*}
        D = \dfrac{\log{3}}{\log{2}}
          \approx & 1.58496.     
    \end{equation*}
\end{example}
\noindent So Sierpenski's Triangle is not 1-dimensional! Or 2-dimensional! Its dimension is somewhere in between!

\begin{exercise}
In your groups, determine the dimensions of the previous three fractals. 
\end{exercise}

\begin{enumerate}
    \item \hspace{2in} \\
    \vspace{2in}
    \item \hspace{2in} \\
    \vspace{2in}
    \item \hspace{2in} \\
\end{enumerate}







